\documentclass[11pt]{article}
\usepackage{graphicx} % Required for inserting images

\usepackage{geometry}
\geometry{
  top=0.5in,
  bottom=1in,
  left=1in,
  right=1in
}

\newcommand{\tabit}{\par\textbullet\phantom{ }}

\renewcommand{\familydefault}{\sfdefault}

\pagenumbering{gobble}

\title{BIOL 633: Fundamentals of Ecological Statistics}
\author{Dr. Samuel V.J. Robinson}
\date{Fall 2023}

\begin{document}
\maketitle

\section*{Outline and Marking}

\large
In this course, we will cover the basics of using the \texttt{R} programming language, along with simple plotting, data organization, and programming techniques. We will also cover the fundamentals of linear modeling before moving onto generalized linear modeling (non-normal distributions), mixed models (i.e. \textit{random effects}), and spatio-temporal effects. Finally, we will discuss how to write about statistical analysis, and will end with short presentations on an analysis of your own datasets (or a simulated dataset, if you haven't collected data yet).

\vspace{1cm}

Proposed marking scheme:
\begin{itemize}
  \item Class participation: 25\%
  \item Final project ``peer review": 25\%
  \begin{itemize}
    \item Create a draft write-up, and provide feedback on your colleagues' work
    \item Write a review document to send to the editor (SR)
  \end{itemize}
  \item Final project write-up: 25\%
  \begin{itemize}
    \item Respond to the feedback from your colleagues, and write a response letter
  \end{itemize}
  \item Final project presentation: 25\%
  \begin{itemize}
    \item Mock committee/proposal meeting: ``what are your main results so far?"
  \end{itemize}
\end{itemize}

\clearpage

\section*{Draft Schedule}

\begin{table}[h]
\centering
\begin{tabular}{p{2cm}|p{4cm}|p{9cm}}

\textbf{Date} & \textbf{Lecture} & \textbf{Learning Outcomes}  \\ \hline

Sep 8 & Intro to R & \tabit Learn R syntax, objects, and basic plotting \tabit Custom functions \tabit Write simple R programs  \\ \hline

Sep 15 & Tidyverse: \texttt{dplyr} \& \texttt{ggplot2}  &  \tabit Principles of graphic design \tabit Introduction to the \texttt{tidyverse} \tabit Data wrangling, filtering, and organization \\ \hline

Sep 22 & Linear Models & \tabit Basic structure and terminology of linear models \tabit Effect sizes, model selection, partial effects plots \tabit Checking model results and output \\ \hline

Sep 29 & Generalized Linear Models (GLMs) & \tabit Common non-normal distributions \tabit GLM fitting and plotting \tabit Model validation, model selection for GLMs \tabit Preliminary models of your own data  \\ \hline

Oct 6 & Mixed effects models & \tabit Random versus fixed effects \tabit Random intercept and slope models \tabit Slope/intercept covariance, hypothesis testing \tabit Plotting of mixed models \\ \hline

Oct 13 & Nonlinear \& Additive models (GAMs) & \tabit Fitting strategies \tabit Generalized additive models (GAMs/``wiggly" models)  \tabit Distributional (non-stationary) models \\ \hline

Oct 20 & Spatiotemporal \& Dynamic models & \tabit Spatial and temporal random effects \tabit Dynamic models (e.g. logistic growth) \\ \hline

Oct 27 & Other topics & \tabit Multivariate models (e.g. community ordination) \tabit R as a GIS (e.g. mapping) \tabit Custom model coding (TMB or Stan) \\ \hline

Nov 3 & Writing & \tabit Structure of scientific papers (IMRaD) \tabit Writing clearly about models \tabit Reading about models critically \\ \hline

Nov 10 & Open work time & \tabit Time for open work on your own models and data \tabit Can work together/ask for help or clarification \\ \hline

Nov 17 & Reading break & Reading break \\ \hline

Nov 24 & Peer review & \tabit Draft write-up due \tabit Show us some of your results! \\ \hline

Dec 1 & Peer review & \tabit Reviews due \\ \hline

Dec 8 & Presentations & \tabit Final presentations \tabit Write-up due \\ 

\end{tabular}
\end{table}

\end{document}
