% Options for packages loaded elsewhere
\PassOptionsToPackage{unicode}{hyperref}
\PassOptionsToPackage{hyphens}{url}
%
\documentclass[
  ignorenonframetext,
  aspectratio=169]{beamer}
\usepackage{pgfpages}
\setbeamertemplate{caption}[numbered]
\setbeamertemplate{caption label separator}{: }
\setbeamercolor{caption name}{fg=normal text.fg}
\beamertemplatenavigationsymbolsempty
% Prevent slide breaks in the middle of a paragraph
\widowpenalties 1 10000
\raggedbottom
\setbeamertemplate{part page}{
  \centering
  \begin{beamercolorbox}[sep=16pt,center]{part title}
    \usebeamerfont{part title}\insertpart\par
  \end{beamercolorbox}
}
\setbeamertemplate{section page}{
  \centering
  \begin{beamercolorbox}[sep=12pt,center]{part title}
    \usebeamerfont{section title}\insertsection\par
  \end{beamercolorbox}
}
\setbeamertemplate{subsection page}{
  \centering
  \begin{beamercolorbox}[sep=8pt,center]{part title}
    \usebeamerfont{subsection title}\insertsubsection\par
  \end{beamercolorbox}
}
\AtBeginPart{
  \frame{\partpage}
}
\AtBeginSection{
  \ifbibliography
  \else
    \frame{\sectionpage}
  \fi
}
\AtBeginSubsection{
  \frame{\subsectionpage}
}
\usepackage{amsmath,amssymb}
\usepackage{iftex}
\ifPDFTeX
  \usepackage[T1]{fontenc}
  \usepackage[utf8]{inputenc}
  \usepackage{textcomp} % provide euro and other symbols
\else % if luatex or xetex
  \usepackage{unicode-math} % this also loads fontspec
  \defaultfontfeatures{Scale=MatchLowercase}
  \defaultfontfeatures[\rmfamily]{Ligatures=TeX,Scale=1}
\fi
\usepackage{lmodern}
\usecolortheme{lily}
\ifPDFTeX\else
  % xetex/luatex font selection
\fi
% Use upquote if available, for straight quotes in verbatim environments
\IfFileExists{upquote.sty}{\usepackage{upquote}}{}
\IfFileExists{microtype.sty}{% use microtype if available
  \usepackage[]{microtype}
  \UseMicrotypeSet[protrusion]{basicmath} % disable protrusion for tt fonts
}{}
\makeatletter
\@ifundefined{KOMAClassName}{% if non-KOMA class
  \IfFileExists{parskip.sty}{%
    \usepackage{parskip}
  }{% else
    \setlength{\parindent}{0pt}
    \setlength{\parskip}{6pt plus 2pt minus 1pt}}
}{% if KOMA class
  \KOMAoptions{parskip=half}}
\makeatother
\usepackage{xcolor}
\newif\ifbibliography
\usepackage{color}
\usepackage{fancyvrb}
\newcommand{\VerbBar}{|}
\newcommand{\VERB}{\Verb[commandchars=\\\{\}]}
\DefineVerbatimEnvironment{Highlighting}{Verbatim}{commandchars=\\\{\}}
% Add ',fontsize=\small' for more characters per line
\usepackage{framed}
\definecolor{shadecolor}{RGB}{248,248,248}
\newenvironment{Shaded}{\begin{snugshade}}{\end{snugshade}}
\newcommand{\AlertTok}[1]{\textcolor[rgb]{0.94,0.16,0.16}{#1}}
\newcommand{\AnnotationTok}[1]{\textcolor[rgb]{0.56,0.35,0.01}{\textbf{\textit{#1}}}}
\newcommand{\AttributeTok}[1]{\textcolor[rgb]{0.13,0.29,0.53}{#1}}
\newcommand{\BaseNTok}[1]{\textcolor[rgb]{0.00,0.00,0.81}{#1}}
\newcommand{\BuiltInTok}[1]{#1}
\newcommand{\CharTok}[1]{\textcolor[rgb]{0.31,0.60,0.02}{#1}}
\newcommand{\CommentTok}[1]{\textcolor[rgb]{0.56,0.35,0.01}{\textit{#1}}}
\newcommand{\CommentVarTok}[1]{\textcolor[rgb]{0.56,0.35,0.01}{\textbf{\textit{#1}}}}
\newcommand{\ConstantTok}[1]{\textcolor[rgb]{0.56,0.35,0.01}{#1}}
\newcommand{\ControlFlowTok}[1]{\textcolor[rgb]{0.13,0.29,0.53}{\textbf{#1}}}
\newcommand{\DataTypeTok}[1]{\textcolor[rgb]{0.13,0.29,0.53}{#1}}
\newcommand{\DecValTok}[1]{\textcolor[rgb]{0.00,0.00,0.81}{#1}}
\newcommand{\DocumentationTok}[1]{\textcolor[rgb]{0.56,0.35,0.01}{\textbf{\textit{#1}}}}
\newcommand{\ErrorTok}[1]{\textcolor[rgb]{0.64,0.00,0.00}{\textbf{#1}}}
\newcommand{\ExtensionTok}[1]{#1}
\newcommand{\FloatTok}[1]{\textcolor[rgb]{0.00,0.00,0.81}{#1}}
\newcommand{\FunctionTok}[1]{\textcolor[rgb]{0.13,0.29,0.53}{\textbf{#1}}}
\newcommand{\ImportTok}[1]{#1}
\newcommand{\InformationTok}[1]{\textcolor[rgb]{0.56,0.35,0.01}{\textbf{\textit{#1}}}}
\newcommand{\KeywordTok}[1]{\textcolor[rgb]{0.13,0.29,0.53}{\textbf{#1}}}
\newcommand{\NormalTok}[1]{#1}
\newcommand{\OperatorTok}[1]{\textcolor[rgb]{0.81,0.36,0.00}{\textbf{#1}}}
\newcommand{\OtherTok}[1]{\textcolor[rgb]{0.56,0.35,0.01}{#1}}
\newcommand{\PreprocessorTok}[1]{\textcolor[rgb]{0.56,0.35,0.01}{\textit{#1}}}
\newcommand{\RegionMarkerTok}[1]{#1}
\newcommand{\SpecialCharTok}[1]{\textcolor[rgb]{0.81,0.36,0.00}{\textbf{#1}}}
\newcommand{\SpecialStringTok}[1]{\textcolor[rgb]{0.31,0.60,0.02}{#1}}
\newcommand{\StringTok}[1]{\textcolor[rgb]{0.31,0.60,0.02}{#1}}
\newcommand{\VariableTok}[1]{\textcolor[rgb]{0.00,0.00,0.00}{#1}}
\newcommand{\VerbatimStringTok}[1]{\textcolor[rgb]{0.31,0.60,0.02}{#1}}
\newcommand{\WarningTok}[1]{\textcolor[rgb]{0.56,0.35,0.01}{\textbf{\textit{#1}}}}
\usepackage{graphicx}
\makeatletter
\def\maxwidth{\ifdim\Gin@nat@width>\linewidth\linewidth\else\Gin@nat@width\fi}
\def\maxheight{\ifdim\Gin@nat@height>\textheight\textheight\else\Gin@nat@height\fi}
\makeatother
% Scale images if necessary, so that they will not overflow the page
% margins by default, and it is still possible to overwrite the defaults
% using explicit options in \includegraphics[width, height, ...]{}
\setkeys{Gin}{width=\maxwidth,height=\maxheight,keepaspectratio}
% Set default figure placement to htbp
\makeatletter
\def\fps@figure{htbp}
\makeatother
\setlength{\emergencystretch}{3em} % prevent overfull lines
\providecommand{\tightlist}{%
  \setlength{\itemsep}{0pt}\setlength{\parskip}{0pt}}
\setcounter{secnumdepth}{-\maxdimen} % remove section numbering
\ifLuaTeX
  \usepackage{selnolig}  % disable illegal ligatures
\fi
\IfFileExists{bookmark.sty}{\usepackage{bookmark}}{\usepackage{hyperref}}
\IfFileExists{xurl.sty}{\usepackage{xurl}}{} % add URL line breaks if available
\urlstyle{same}
\hypersetup{
  pdftitle={Introduction to R},
  pdfauthor={Samuel Robinson, Ph.D.},
  hidelinks,
  pdfcreator={LaTeX via pandoc}}

\title{Introduction to R}
\subtitle{``How do you turn this thing on?''}
\author{Samuel Robinson, Ph.D.}
\date{Sep.~4 2023}

\begin{document}
\frame{\titlepage}

\begin{frame}{Motivation}
\protect\hypertarget{motivation}{}
\begin{itemize}[<+->]
\tightlist
\item
  ``Why do I need to learn R?''

  \begin{itemize}[<+->]
  \tightlist
  \item
    Free, powerful, and very common
  \item
    Interfaces with other languages (e.g.~C++), and can help you learn
    other languages
  \end{itemize}
\item
  ``What is R good at?''

  \begin{itemize}[<+->]
  \tightlist
  \item
    Displaying data, running models, and processing data*
  \item
    Writing presentations and papers
  \item
    \emph{Keeping a record of what you've done}
  \end{itemize}
\item
  ``What is R bad at?''

  \begin{itemize}[<+->]
  \tightlist
  \item
    No point-and-click interface; simple things can take more time
  \item
    Can be slow if datasets are large*
  \end{itemize}
\item
  I am not here to teach you programming, but some basic techniques are
  useful
\end{itemize}
\end{frame}

\begin{frame}{Console and Scripts}
\protect\hypertarget{console-and-scripts}{}
\begin{itemize}[<+->]
\tightlist
\item
  The \textbf{Console} is the main input into R (where you tell it to do
  things)
\item
  \textbf{Scripts} are lists of commands that get passed into the
  console
\item
  If you're using RStudio, 2 of the 4 panes will be dedicated to the
  console and scripts
\end{itemize}
\end{frame}

\begin{frame}{RStudio GUI}
\protect\hypertarget{rstudio-gui}{}
\begin{columns}

\column{0.5\textwidth}

![](screenshot.png){width=50%}

\column{0.5\textwidth}

My (custom) RStudio setup

\end{columns}
\end{frame}

\begin{frame}{Objects and Functions}
\protect\hypertarget{objects-and-functions}{}
\begin{itemize}[<+->]
\tightlist
\item
  Everything in R is either an \textbf{Object} or a \textbf{Function}.
  All must have a unique name, or else the \emph{Steve Problem}* occurs.
\item
  Some common \textbf{objects} (things stored in memory):

  \begin{itemize}[<+->]
  \tightlist
  \item
    Strings, Logicals, \& Numerics
  \item
    Vectors \& Matrices
  \item
    Dataframes \& Lists*
  \end{itemize}
\item
  Some common \textbf{functions} (things done to objects):

  \begin{itemize}[<+->]
  \tightlist
  \item
    \emph{mean}, \emph{sd}, \emph{median}, \emph{quantile}, \emph{c},
    \emph{paste}
  \item
    \emph{plot}, \emph{summary}

    \begin{itemize}[<+->]
    \tightlist
    \item
      these are polymorphic functions: they do different things to
      different types of objects
    \end{itemize}
  \item
    Control flow - \emph{if} and \emph{for}
  \end{itemize}
\end{itemize}
\end{frame}

\begin{frame}[fragile]{Objects}
\protect\hypertarget{objects}{}
\begin{itemize}
\tightlist
\item
  Let's make some objects. These are all single objects:
\end{itemize}

\footnotesize

\begin{quote}
\begin{Shaded}
\begin{Highlighting}[]
\NormalTok{myString }\OtherTok{\textless{}{-}} \StringTok{"Hello world"} \CommentTok{\#A string object}
\NormalTok{myNumeric }\OtherTok{\textless{}{-}} \DecValTok{12345} \CommentTok{\#A numeric object}
\NormalTok{myLogical }\OtherTok{\textless{}{-}} \ConstantTok{TRUE} \CommentTok{\#A logical object}
\end{Highlighting}
\end{Shaded}
\end{quote}

\normalsize

\begin{itemize}
\tightlist
\item
  These are objects joined into a \emph{vector}, joined by the function
  c (concatenate):
\end{itemize}

\footnotesize

\begin{quote}
\begin{Shaded}
\begin{Highlighting}[]
\NormalTok{myStringVec }\OtherTok{\textless{}{-}} \FunctionTok{c}\NormalTok{(}\StringTok{"I like pie"}\NormalTok{,}\StringTok{"I like cake"}\NormalTok{,}\StringTok{"I like anything you bake"}\NormalTok{)}
\NormalTok{myNumericVec }\OtherTok{\textless{}{-}} \FunctionTok{c}\NormalTok{(}\DecValTok{1}\NormalTok{,}\DecValTok{2}\NormalTok{,}\DecValTok{3}\NormalTok{,}\DecValTok{4}\NormalTok{,}\DecValTok{5}\NormalTok{)}
\NormalTok{myLogicalVec }\OtherTok{\textless{}{-}} \FunctionTok{c}\NormalTok{(}\ConstantTok{TRUE}\NormalTok{,}\ConstantTok{TRUE}\NormalTok{,}\ConstantTok{FALSE}\NormalTok{,}\ConstantTok{TRUE}\NormalTok{,}\ConstantTok{FALSE}\NormalTok{,}\ConstantTok{FALSE}\NormalTok{)}
\end{Highlighting}
\end{Shaded}
\end{quote}

\normalsize

\begin{itemize}
\tightlist
\item
  How long are each of these vectors?
\end{itemize}

\footnotesize

\begin{quote}
\begin{Shaded}
\begin{Highlighting}[]
\NormalTok{howLong }\OtherTok{\textless{}{-}} \FunctionTok{c}\NormalTok{(}\FunctionTok{length}\NormalTok{(myStringVec), }\FunctionTok{length}\NormalTok{(myNumericVec), }\FunctionTok{length}\NormalTok{(myLogicalVec))}
\NormalTok{howLong }\CommentTok{\#This executes the \textasciigrave{}print\textasciigrave{} command on \textasciigrave{}howLong\textasciigrave{}}
\end{Highlighting}
\end{Shaded}

\begin{verbatim}
## [1] 3 5 6
\end{verbatim}
\end{quote}

\normalsize
\end{frame}

\begin{frame}[fragile]{Vectors}
\protect\hypertarget{vectors}{}
\begin{itemize}[<+->]
\tightlist
\item
  How do I get stuff out of the vectors I just made?
\end{itemize}

\footnotesize

\begin{quote}
\begin{Shaded}
\begin{Highlighting}[]
\NormalTok{myStringVec }\CommentTok{\#Here\textquotesingle{}s what\textquotesingle{}s inside the whole thing}
\end{Highlighting}
\end{Shaded}

\begin{verbatim}
## [1] "I like pie"               "I like cake"             
## [3] "I like anything you bake"
\end{verbatim}

\begin{Shaded}
\begin{Highlighting}[]
\NormalTok{myStringVec[}\DecValTok{1}\NormalTok{] }\CommentTok{\#Uses a single numeric}
\end{Highlighting}
\end{Shaded}

\begin{verbatim}
## [1] "I like pie"
\end{verbatim}

\begin{Shaded}
\begin{Highlighting}[]
\NormalTok{myStringVec[}\FunctionTok{c}\NormalTok{(}\DecValTok{2}\NormalTok{,}\DecValTok{3}\NormalTok{)] }\CommentTok{\#Uses a vector of numerics}
\end{Highlighting}
\end{Shaded}

\begin{verbatim}
## [1] "I like cake"              "I like anything you bake"
\end{verbatim}

\begin{Shaded}
\begin{Highlighting}[]
\NormalTok{myStringVec[}\FunctionTok{c}\NormalTok{(}\ConstantTok{TRUE}\NormalTok{,}\ConstantTok{FALSE}\NormalTok{,}\ConstantTok{TRUE}\NormalTok{)] }\CommentTok{\#Uses a logical vector of same length}
\end{Highlighting}
\end{Shaded}

\begin{verbatim}
## [1] "I like pie"               "I like anything you bake"
\end{verbatim}
\end{quote}

\normalsize
\end{frame}

\begin{frame}[fragile]{Matrices}
\protect\hypertarget{matrices}{}
\begin{itemize}
\tightlist
\item
  Matrices are rectangular structures that hold values inside them:
\end{itemize}

\footnotesize

\begin{quote}
\begin{Shaded}
\begin{Highlighting}[]
\NormalTok{(myMatrix }\OtherTok{\textless{}{-}} \FunctionTok{matrix}\NormalTok{(}\DecValTok{1}\SpecialCharTok{:}\DecValTok{9}\NormalTok{,}\AttributeTok{ncol=}\DecValTok{3}\NormalTok{))}
\end{Highlighting}
\end{Shaded}

\begin{verbatim}
##      [,1] [,2] [,3]
## [1,]    1    4    7
## [2,]    2    5    8
## [3,]    3    6    9
\end{verbatim}
\end{quote}

\normalsize

\begin{itemize}
\tightlist
\item
  Matrices are indexed by \emph{rows} and \emph{columns} (in that
  order):
\end{itemize}

\footnotesize

\begin{quote}
\begin{Shaded}
\begin{Highlighting}[]
\NormalTok{myMatrix[}\DecValTok{1}\NormalTok{,}\DecValTok{3}\NormalTok{] }\CommentTok{\#1st row, 3rd col}
\end{Highlighting}
\end{Shaded}

\begin{verbatim}
## [1] 7
\end{verbatim}

\begin{Shaded}
\begin{Highlighting}[]
\NormalTok{myMatrix[,}\DecValTok{3}\NormalTok{] }\CommentTok{\#All rows, 3rd column}
\end{Highlighting}
\end{Shaded}

\begin{verbatim}
## [1] 7 8 9
\end{verbatim}
\end{quote}

\normalsize
\end{frame}

\begin{frame}[fragile]{Dataframes}
\protect\hypertarget{dataframes}{}
\begin{itemize}[<+->]
\tightlist
\item
  Dataframes look similar to matrices, but can hold different data types
  in each column:
\end{itemize}

\footnotesize

\begin{Shaded}
\begin{Highlighting}[]
\CommentTok{\#Each column has a unique name, and must be the same length}
\NormalTok{myDF }\OtherTok{\textless{}{-}} \FunctionTok{data.frame}\NormalTok{(}\AttributeTok{stringCol=}\NormalTok{myStringVec, }\AttributeTok{numCol=}\NormalTok{myNumericVec[}\FunctionTok{c}\NormalTok{(}\DecValTok{1}\SpecialCharTok{:}\DecValTok{3}\NormalTok{)], }
                   \AttributeTok{logCol=}\NormalTok{myLogicalVec[}\FunctionTok{c}\NormalTok{(}\DecValTok{1}\SpecialCharTok{:}\DecValTok{3}\NormalTok{)])}
\NormalTok{myDF}
\end{Highlighting}
\end{Shaded}

\begin{verbatim}
##                  stringCol numCol logCol
## 1               I like pie      1   TRUE
## 2              I like cake      2   TRUE
## 3 I like anything you bake      3  FALSE
\end{verbatim}

\begin{Shaded}
\begin{Highlighting}[]
\FunctionTok{summary}\NormalTok{(myDF) }\CommentTok{\#This function summarizes each column}
\end{Highlighting}
\end{Shaded}

\begin{verbatim}
##   stringCol             numCol      logCol       
##  Length:3           Min.   :1.0   Mode :logical  
##  Class :character   1st Qu.:1.5   FALSE:1        
##  Mode  :character   Median :2.0   TRUE :2        
##                     Mean   :2.0                  
##                     3rd Qu.:2.5                  
##                     Max.   :3.0
\end{verbatim}

\normalsize
\end{frame}

\begin{frame}[fragile]{Lists}
\protect\hypertarget{lists}{}
\begin{itemize}[<+->]
\item
  Lists look similar to vectors, but can hold anything in each slot,
  including other lists.
\item
  LOTS of things in R (e.g.~model output) are specially-structured lists
  at their core
\end{itemize}

\footnotesize

\begin{Shaded}
\begin{Highlighting}[]
\NormalTok{myList }\OtherTok{\textless{}{-}} \FunctionTok{list}\NormalTok{(}\AttributeTok{stringSlot=}\NormalTok{myStringVec, }\AttributeTok{numSlot=}\NormalTok{myNumericVec, }
                   \AttributeTok{logSlot=}\NormalTok{myLogicalVec, }\AttributeTok{dfSlot =}\NormalTok{ myDF)}
\NormalTok{myList}
\end{Highlighting}
\end{Shaded}

\begin{verbatim}
## $stringSlot
## [1] "I like pie"               "I like cake"             
## [3] "I like anything you bake"
## 
## $numSlot
## [1] 1 2 3 4 5
## 
## $logSlot
## [1]  TRUE  TRUE FALSE  TRUE FALSE FALSE
## 
## $dfSlot
##                  stringCol numCol logCol
## 1               I like pie      1   TRUE
## 2              I like cake      2   TRUE
## 3 I like anything you bake      3  FALSE
\end{verbatim}

\normalsize
\end{frame}

\begin{frame}[fragile]{Accessing Dataframes}
\protect\hypertarget{accessing-dataframes}{}
\begin{itemize}[<+->]
\tightlist
\item
  Dataframes can be accessed numerically, by their name slots, or with a
  mixture of the two:
\end{itemize}

\footnotesize

\begin{Shaded}
\begin{Highlighting}[]
\NormalTok{myDF[}\DecValTok{1}\NormalTok{,}\DecValTok{2}\NormalTok{]}
\end{Highlighting}
\end{Shaded}

\begin{verbatim}
## [1] 1
\end{verbatim}

\begin{Shaded}
\begin{Highlighting}[]
\NormalTok{myDF}\SpecialCharTok{$}\NormalTok{numCol }\CommentTok{\#This gets all of the column "numCol"}
\end{Highlighting}
\end{Shaded}

\begin{verbatim}
## [1] 1 2 3
\end{verbatim}

\begin{Shaded}
\begin{Highlighting}[]
\NormalTok{myDF[}\DecValTok{1}\NormalTok{,}\StringTok{"numCol"}\NormalTok{]}
\end{Highlighting}
\end{Shaded}

\begin{verbatim}
## [1] 1
\end{verbatim}

\normalsize
\end{frame}

\begin{frame}[fragile]{Accessing Lists}
\protect\hypertarget{accessing-lists}{}
\begin{itemize}[<+->]
\tightlist
\item
  Similarly, lists can be accessed numerically (see below), or by their
  name slots:
\end{itemize}

\footnotesize

\begin{Shaded}
\begin{Highlighting}[]
\NormalTok{myList[[}\DecValTok{2}\NormalTok{]] }\CommentTok{\#Needs 2 square brackets to isolate object}
\end{Highlighting}
\end{Shaded}

\begin{verbatim}
## [1] 1 2 3 4 5
\end{verbatim}

\begin{Shaded}
\begin{Highlighting}[]
\NormalTok{myList[[}\StringTok{\textquotesingle{}numSlot\textquotesingle{}}\NormalTok{]]}
\end{Highlighting}
\end{Shaded}

\begin{verbatim}
## [1] 1 2 3 4 5
\end{verbatim}

\begin{Shaded}
\begin{Highlighting}[]
\NormalTok{myList}\SpecialCharTok{$}\NormalTok{numSlot}
\end{Highlighting}
\end{Shaded}

\begin{verbatim}
## [1] 1 2 3 4 5
\end{verbatim}

\begin{Shaded}
\begin{Highlighting}[]
\NormalTok{myList[[}\DecValTok{4}\NormalTok{]][,}\DecValTok{3}\NormalTok{] }\CommentTok{\#Same as myList$dfSlot$logCol}
\end{Highlighting}
\end{Shaded}

\begin{verbatim}
## [1]  TRUE  TRUE FALSE
\end{verbatim}

\normalsize
\end{frame}

\begin{frame}[fragile]{Functions}
\protect\hypertarget{functions}{}
\begin{itemize}[<+->]
\tightlist
\item
  Functions take objects as \textbf{arguments} (input) and return other
  \textbf{objects} (output)
\end{itemize}

\footnotesize

\begin{Shaded}
\begin{Highlighting}[]
\NormalTok{myNumericVec}
\NormalTok{meanVec }\OtherTok{\textless{}{-}} \FunctionTok{mean}\NormalTok{(myNumericVec) }\CommentTok{\#Arithmetic mean (average)}
\NormalTok{sdVec }\OtherTok{\textless{}{-}} \FunctionTok{sd}\NormalTok{(myNumericVec) }\CommentTok{\#Standard deviation (sqrt(variance))}
\NormalTok{meanSdVec }\OtherTok{\textless{}{-}} \FunctionTok{c}\NormalTok{(meanVec,sdVec) }\CommentTok{\#Joins mean and SD into a vector}
\NormalTok{meanSdVec}
\NormalTok{?median }\CommentTok{\#If you can\textquotesingle{}t remember how a command works, use "?" to access the help files}
\end{Highlighting}
\end{Shaded}

\normalsize
\end{frame}

\begin{frame}[fragile]{Homemade Functions}
\protect\hypertarget{homemade-functions}{}
\begin{itemize}[<+->]
\tightlist
\item
  You can make your own functions! This is useful if you have to do the
  same thing to many different input objects.
\end{itemize}

\footnotesize

\begin{Shaded}
\begin{Highlighting}[]
\NormalTok{myFun }\OtherTok{\textless{}{-}} \ControlFlowTok{function}\NormalTok{(input)\{ }\CommentTok{\#Takes a vector of numbers}
\NormalTok{  A }\OtherTok{\textless{}{-}} \FunctionTok{mean}\NormalTok{(input) }\CommentTok{\#Take the mean of INPUT}
\NormalTok{  B }\OtherTok{\textless{}{-}} \FunctionTok{sd}\NormalTok{(input) }\CommentTok{\#Take the SD of INPUT}
\NormalTok{  C }\OtherTok{\textless{}{-}} \FunctionTok{c}\NormalTok{(A,B) }\CommentTok{\#Join A and B into a vector C}
  \FunctionTok{return}\NormalTok{(C) }\CommentTok{\#Return (output) C, then end the function}
\NormalTok{\}}
\FunctionTok{myFun}\NormalTok{(myNumericVec) }\CommentTok{\#Same as previous slide}
\end{Highlighting}
\end{Shaded}

\begin{verbatim}
## [1] 3.000000 1.581139
\end{verbatim}

\begin{Shaded}
\begin{Highlighting}[]
\FunctionTok{myFun}\NormalTok{(myLogicalVec) }\CommentTok{\#Logical vector is converted to 1 (TRUE) and 0 (FALSE)}
\end{Highlighting}
\end{Shaded}

\begin{verbatim}
## [1] 0.5000000 0.5477226
\end{verbatim}

\normalsize

\begin{itemize}[<+->]
\tightlist
\item
  The objects inside of functions (A,B,C in the one above) disappear
  after the function runs. However, functions can see objects in the
  outer environment, so beware of the Steve Problem*
\end{itemize}
\end{frame}

\begin{frame}[fragile]{Plotting}
\protect\hypertarget{plotting}{}
\begin{itemize}[<+->]
\tightlist
\item
  The \emph{plot} command is useful for quickly looking at sets of data.
  The following CO2 dataset is built-in to R. To see others, type data()
\end{itemize}

\footnotesize

\begin{Shaded}
\begin{Highlighting}[]
\CommentTok{\#Makes a plot of the uptake (y) and conc columns of CO2, and customize axis labels}
\FunctionTok{plot}\NormalTok{(}\AttributeTok{x =}\NormalTok{ CO2}\SpecialCharTok{$}\NormalTok{conc, }\AttributeTok{y =}\NormalTok{ CO2}\SpecialCharTok{$}\NormalTok{uptake, }\AttributeTok{xlab =} \StringTok{\textquotesingle{}Concentration (ppm)\textquotesingle{}}\NormalTok{, }\AttributeTok{ylab =} \StringTok{\textquotesingle{}Uptake (g)\textquotesingle{}}\NormalTok{)}
\end{Highlighting}
\end{Shaded}

\includegraphics{01-Lecture_files/figure-beamer/unnamed-chunk-13-1.pdf}

\normalsize
\end{frame}

\begin{frame}[fragile]{Plotting (cont.)}
\protect\hypertarget{plotting-cont.}{}
\begin{itemize}[<+->]
\tightlist
\item
  The boxplot command can summarize \emph{continuous} and
  \emph{categorical} data
\end{itemize}

\footnotesize

\begin{Shaded}
\begin{Highlighting}[]
\CommentTok{\#Boxplot uses a formula rather than x,y vectors }
\CommentTok{\#Note: plot can also use formulas in lieu of x,y}
\FunctionTok{boxplot}\NormalTok{(CO2}\SpecialCharTok{$}\NormalTok{uptake }\SpecialCharTok{\textasciitilde{}}\NormalTok{ CO2}\SpecialCharTok{$}\NormalTok{Type, }\CommentTok{\#Formula where uptake depends on (\textasciitilde{}) groups}
        \AttributeTok{xlab=}\StringTok{\textquotesingle{}Type\textquotesingle{}}\NormalTok{, }\AttributeTok{ylab=}\StringTok{\textquotesingle{}Uptake (g)\textquotesingle{}}\NormalTok{)}
\end{Highlighting}
\end{Shaded}

\includegraphics{01-Lecture_files/figure-beamer/unnamed-chunk-14-1.pdf}

\normalsize
\end{frame}

\begin{frame}[fragile]{Summary statistics}
\protect\hypertarget{summary-statistics}{}
\begin{itemize}[<+->]
\tightlist
\item
  Often we want to get the mean of one columns, but split it up by other
  things in the dataframe. Using the CO2 plant example, how does
  \emph{uptake} differ between \emph{Type}?
\end{itemize}

\footnotesize

\begin{Shaded}
\begin{Highlighting}[]
\CommentTok{\#Split up uptake by Type and Treatment, then take the mean}
\FunctionTok{tapply}\NormalTok{(CO2}\SpecialCharTok{$}\NormalTok{uptake,}\FunctionTok{list}\NormalTok{(CO2}\SpecialCharTok{$}\NormalTok{Type,CO2}\SpecialCharTok{$}\NormalTok{Treatment),mean)}
\end{Highlighting}
\end{Shaded}

\begin{verbatim}
##             nonchilled  chilled
## Quebec        35.33333 31.75238
## Mississippi   25.95238 15.81429
\end{verbatim}

\normalsize

\begin{itemize}[<+->]
\tightlist
\item
  It's annoying and repetitive to type ``CO2'' over and over again. You
  can use \emph{with} to avoid this (avoid using \emph{attach}):
\end{itemize}

\footnotesize

\begin{Shaded}
\begin{Highlighting}[]
\CommentTok{\#Runs command inside the name space of the CO2 object}
\FunctionTok{with}\NormalTok{(CO2, }\FunctionTok{tapply}\NormalTok{(uptake,}\FunctionTok{list}\NormalTok{(Type,Treatment),sd))}
\end{Highlighting}
\end{Shaded}

\begin{verbatim}
##             nonchilled  chilled
## Quebec        9.596371 9.644823
## Mississippi   7.402136 4.058976
\end{verbatim}

\normalsize
\end{frame}

\begin{frame}[fragile]{if statements}
\protect\hypertarget{if-statements}{}
\begin{itemize}[<+->]
\tightlist
\item
  R can be told to do things only \emph{if} certain conditions apply.
  This is useful inside of functions for error handling:
\end{itemize}

\footnotesize

\begin{Shaded}
\begin{Highlighting}[]
\NormalTok{myFun2 }\OtherTok{\textless{}{-}} \ControlFlowTok{function}\NormalTok{(x)\{}
\NormalTok{  xClass }\OtherTok{\textless{}{-}} \FunctionTok{class}\NormalTok{(x) }\CommentTok{\#What class is x? (Numeric, character, boolean)}
  
  \ControlFlowTok{if}\NormalTok{(xClass}\SpecialCharTok{==}\StringTok{\textquotesingle{}character\textquotesingle{}}\NormalTok{)\{ }\CommentTok{\#== means "are these things equal"?}
    \FunctionTok{return}\NormalTok{(}\StringTok{\textquotesingle{}This is a string\textquotesingle{}}\NormalTok{) }\CommentTok{\#If x is a character, returns a message}
\NormalTok{  \} }\ControlFlowTok{else}\NormalTok{ \{}
    \FunctionTok{return}\NormalTok{(}\FunctionTok{mean}\NormalTok{(x)) }\CommentTok{\#If x isn\textquotesingle{}t a character, returns the mean of x}
\NormalTok{  \}}
\NormalTok{\}}
\FunctionTok{myFun2}\NormalTok{(myStringVec)}
\end{Highlighting}
\end{Shaded}

\begin{verbatim}
## [1] "This is a string"
\end{verbatim}

\begin{Shaded}
\begin{Highlighting}[]
\FunctionTok{myFun2}\NormalTok{(myNumericVec)}
\end{Highlighting}
\end{Shaded}

\begin{verbatim}
## [1] 3
\end{verbatim}

\normalsize
\end{frame}

\begin{frame}[fragile]{for loops}
\protect\hypertarget{for-loops}{}
\begin{itemize}[<+->]
\tightlist
\item
  R can be told to do things \emph{repeatedly}, using an index:
\end{itemize}

\footnotesize

\begin{Shaded}
\begin{Highlighting}[]
\NormalTok{classVec }\OtherTok{\textless{}{-}} \FunctionTok{rep}\NormalTok{(}\StringTok{\textquotesingle{}\textquotesingle{}}\NormalTok{,}\FunctionTok{length}\NormalTok{(myList)) }\CommentTok{\#Storage vector}

\CommentTok{\#i will take on values 1 to 4, each time the loop repeats}
\ControlFlowTok{for}\NormalTok{(i }\ControlFlowTok{in} \DecValTok{1}\SpecialCharTok{:}\FunctionTok{length}\NormalTok{(myList))\{}
  
  \CommentTok{\#ith slot of classVec becomes class from ith slot of myList}
\NormalTok{  classVec[i] }\OtherTok{\textless{}{-}} \FunctionTok{class}\NormalTok{(myList[[i]]) }
\NormalTok{\}}
\NormalTok{classVec}
\end{Highlighting}
\end{Shaded}

\begin{verbatim}
## [1] "character"  "numeric"    "logical"    "data.frame"
\end{verbatim}

\normalsize
\end{frame}

\begin{frame}[fragile]{Reading csv files}
\protect\hypertarget{reading-csv-files}{}
\begin{itemize}[<+->]
\tightlist
\item
  One very common practice is to read in your own data from a csv file.
  Excel files can be read in directly, but present other problems.
\end{itemize}

\footnotesize

\begin{Shaded}
\begin{Highlighting}[]
\NormalTok{testDat }\OtherTok{\textless{}{-}} \FunctionTok{read.csv}\NormalTok{(}\StringTok{\textquotesingle{}test\_results.csv\textquotesingle{}}\NormalTok{)}
\FunctionTok{head}\NormalTok{(testDat) }\CommentTok{\#head shows only first 6 rows of dataframe}
\end{Highlighting}
\end{Shaded}

\begin{verbatim}
##   Concentration Treatment Lab.Member Time.of.Day
## 1           2.9   Control       Will     Morning
## 2           3.2   Control       Will     Morning
## 3           3.6   Control       Will     Morning
## 4           5.6         A       Will     Morning
## 5           6.8         A       Will     Morning
## 6           7.0         A       Will     Morning
\end{verbatim}

\normalsize

\begin{itemize}[<+->]
\tightlist
\item
  R can't handle spaces or other special characters in the column
  headers (replaces them with periods). It also tries to guess the
  proper data type for each column, but sometimes gets this wrong.
\end{itemize}
\end{frame}

\begin{frame}{A challenger approaches!}
\protect\hypertarget{a-challenger-approaches}{}
\begin{itemize}[<+->]
\item
  Your supervisor has just given you a dataset
  (\emph{test\_results.csv}) recorded by two undergrads. However, these
  undergrads were in a big hurry, and have made some mistakes.

  \begin{itemize}[<+->]
  \item
    Read the \emph{csv} file and fix any mistakes. Bonus if you do this
    without using Excel!
  \item
    Plot the concentration data by treatment group, then plot it for
    each undergrad. Does there look like much of a difference?
  \item
    Some useful commands: \textbf{boxplot}, \textbf{grepl},
    \textbf{gsub}, \textbf{as.numeric}
  \end{itemize}
\item
  Make R sing ``99 bottles of beer on the wall''. Bonus points if you
  can use a function that you built!

  \begin{itemize}[<+->]
  \item
    Some useful commands: \textbf{for}, \textbf{if}, \textbf{paste},
    \textbf{print}
  \item
    \href{https://en.wikipedia.org/wiki/99_Bottles_of_Beer}{Don't know
    the song?}
  \end{itemize}
\end{itemize}
\end{frame}

\end{document}
