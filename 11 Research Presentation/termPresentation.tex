\documentclass[11pt]{article}
\usepackage{graphicx} % Required for inserting images
% \usepackage{multicol} %Multiple columns/rows in tables
\usepackage{geometry}
\geometry{
  top=0.5in,
  bottom=1in,
  left=1in,
  right=1in
}

\newcommand{\tabit}{\scriptsize\par\textbullet\phantom{ }}

\renewcommand{\familydefault}{\sfdefault}

\pagenumbering{gobble}

\title{BIOL 633: Term Presentation}
\author{Dr. Samuel V.J. Robinson}
\date{Fall 2023}

\begin{document}
\maketitle

\section*{Research Talks}

In this segment of the course, we will practice another valuable skill for you career as a scientist: a research talk. Research talks are usually similar in form to IMRaD manuscripts, but are far more condensed. Therefore, you must make crucial decisions about \emph{exactly} what your audience needs to know, as there is very little time in a presentation to get side-tracked on minor details. The target audience for your talk will obviously depend on the venue you're presenting at, but one of my rules-of-thumb for creating a presentation is: ``\emph{could my grandma understand this presentation}"? In others words, would an interested (but not formally trained) person understand the problem and my main findings? This requires a proper amount of set-up, a clear statement of your research goals, good figure and slide design, and an engaging teaching style to do well. Remember: you're telling a story about your own research! 

Overall, I would encourage you to think of past presentation you've seen that was \emph{effective and memorable}, and was also \emph{outside of your particular sub-field}:
\begin{itemize}
\item What made the presentation easy to follow and interesting?
\item What storytelling techniques did the speaker use? e.g. character development (``Let me introduce you to my study system...") or plot twists (``We actually found the opposite pattern!")
\item What did their figures look like? Did they use animations or small videos?
\item How did the speaker address the audience? 
\end{itemize}

\subsection*{Note on public speaking}

It is scary to get up in front of a room full of people and talk on your own! Practice is the best way to alleviate this, so ask your roommates/labmates/family to let you practice your talk beforehand. Unfortunately, some people get extremely anxious about any kind of public speaking, even after lots practice and preparation. My goal here is not to publically terrify you, but prepare you for future occasions when you'll have to communicate your science in a talk. Please let me know if this applies to you, and we can talk about potential solutions.

\newpage

\section*{Expectatations}

\begin{table}[h!]
\centering
\begin{tabular}{p{4cm}|p{11cm}}

\textbf{Section} & \textbf{Requirements}  \\ \hline

Background & \textbf{Provides enough background information for a novice to understand the problem} \tabit What key theories are you trying to test? \tabit What problems are you trying to solve? \tabit Why does any of this matter?  \\ \hline

Questions \& Hypotheses & \textbf{Hypotheses and research questions are clearly introduced} \\ \hline

Methods & \textbf{Data collection and analysis are concisely stated} \tabit Methods sections in presentations have far fewer details than those in manuscripts \tabit Some people will put a condensed form of their data or models in this section, and then have ``backup slides" after the end of the presentation that they use only to answer questions from the audience \\ \hline

Results \& Implications & \textbf{The results of your analysis are clearly explained, as well as implications of these results} \tabit Results are related back to the original hypotheses and predictions \tabit \textbf{For proposal presentations:} this section could give a sneak-peek of your upcoming analysis or show results from a simulated/similar dataset, and then finish with potential implications \\ \hline

Timing & \textbf{Presentation is no longer than 13 minutes, with 2 minutes provided for questions} \\ \hline

Visual Presentation & \textbf{Slides and figures are well-designed, and effectively used by the presenter}
\tabit Text on slides (when present) is minimal, and uses a large font size
\tabit Figures are easy to understand, and the author uses them well
\tabit For dynamic processes, consider using a small animation or video

\\ \hline

Teaching Style & \textbf{Speaker's speaking and presentation style is clear and effective} 
\tabit Face your audience directly, and avoid reading text off of slides 
\tabit Questions or thought experiments can direct the audience's attention to your reasoning process 
\tabit Humour is optional, but it can relax people and make the presentation more memorable 
\tabit Be honest when answering questions: e.g. "I don't know, but I think it might be something like..." 
\\

\end{tabular}
\end{table}
 
\end{document}
