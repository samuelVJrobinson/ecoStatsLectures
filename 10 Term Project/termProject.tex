\documentclass[11pt]{article}
\usepackage{graphicx} % Required for inserting images

\usepackage{geometry}
\geometry{
  top=0.5in,
  bottom=1in,
  left=1in,
  right=1in
}

\newcommand{\tabit}{\par\textbullet\phantom{ }}

\renewcommand{\familydefault}{\sfdefault}

\pagenumbering{gobble}

\title{BIOL 633: Term Project}
\author{Dr. Samuel V.J. Robinson}
\date{Fall 2023}

\begin{document}
\maketitle

\section*{Preamble}

\large
At this point in the term, we have learned about some common statistical approaches to different problems in ecology and evolution, and will now turn our efforts to putting this into practice. In this section we will gain experience in: 

\begin{itemize}
\item Creating research proposals OR IMRaD-style manuscripts 
\item Providing critical, useful feedback in the form of a ``mock" peer-review
\item Responding to feedback and further refining your manuscripts or proposals
\item Presenting your research to an audience of peers in a conference-style presentation
\end{itemize}

Doing statistics and data analysis as a working scientist is not simply a matter of fitting the ``correct" model or set of analyses and then pasting your ensuing results into a document. Like most academic disciplines, it is a matter of convincing your audience using a combination of word-smithing, modeling framework or mathematics (where needed), and graphic design. Even the most brilliant data collection and analysis is worth little if nobody can read or understand it! However, a lucid writing style and attractive figures are (hopefully) not everything: you must also convince your audience that the data you collected is a) appropriate to answer your particular question and that b) your methods of analysis were appropriate.

\section*{Proposal or Manuscript}

Depending on the timing of your particular degree program, you may choose to write either a \emph{proposal} or a \emph{manuscript}. If you've recently submitted your proposal, you may choose to do a more analytical version of it, but I would encourage you to be as forward-looking as possible, as the goal of this assignment is ultimately to produce material that can be used to further your scientific career.

A \textbf{proposal} must:
\begin{itemize}
  \item Explain and justify your research topic
  \begin{itemize}
    \item Introduce the reader to your research topic, and place it within the overall scope of your discipline
    \item Reveal the current problems or gaps in understanding: what is not known, and why might this be?
    \item Explain why this problem matters to us. Is the answer to this problem theoretically or practically important? 
    \item Introduce your research goals to the reader in relation to the research gaps you've identified. Often this can be framed as your \emph{hypothesis}, or put in another way, "how you think the world works". What are the causal agents you identify?
    \item Make \emph{predictions} about what you're likely to find. If you find a certain results (positive, negative, neutral), will this support your original hypothesis, or lend support to an alternative one? Simple figures are often useful for this.
  \end{itemize}
  \item Create a data collection plan
  \begin{itemize}
    \item Outline your methods of data collection 
    \item Types of experimental controls/treatments that you plan to use
    \item Possible field work locations or setups
    \item Other sources of data (e.g. a literature review in the case of a meta-analysis)
    \item Identify what permissions or ethics approvals are required
  \end{itemize}
  \item Create a data analysis plan
  \begin{itemize}
    \item What types of models or analysis will you use to answer your questions?
    \item Of the data that you collect, what are the \emph{dependent} and \emph{independent} variables?
    \begin{itemize} 
      \item If you're dealing primarily with observational data, defend your causal assumptions (i.e. why does X cause Y, rather than Y causing X?)
    \end{itemize}
    \item How are your models structured? R model formulas or mathematical structures can be helpful
    \item What parameters or other output will you use to idenfity support for your hypotheses? What types of tests will you use to identify strong ("significant") effects or negligible ones?
    \item How might these results look in a figure? This can be related to your earlier "predictions" figures
  \end{itemize}
  \item Identify an action plan, potential collaborators, timeline, and budget (\emph{not required for this assignment})
\end{itemize}

An IMRaD (Introduction, Methods, Results, and Discussion) manuscript will be similar to the proposal in many respects, in particular with reference to introducing and justifying your research topic and approach, but in this case will be referring to things that you've already done. These manuscripts follow this general form:

\begin{itemize}
\item Introduction
\begin{itemize}
  \item See "Explain and justify your research topic" above
\end{itemize}
\item Methods
\begin{itemize}
  \item See "Create a data collection plan" and "Create a data analysis plan" above
\end{itemize}
\item Results
\begin{itemize}
  \item Describe the data you collected, results of your analysis, and how it relates to your original hypothesis and predictions 
  \item Decide which figures and tables to include in your manuscript. A rule of thumb for an "average" manuscript might be 3 figures and 2 tables. Extra supporting information can go into Supplemental/Appendix (see below) 
\end{itemize}
\item Discussion
\begin{itemize}
  \item See "Create a data collection plan" and "Create a data analysis plan" above
  \item Decide which figures and tables to include in your manuscript. A rule of thumb for an "average" manuscript might be 3 figures and 2 tables. Extra supporting information can go into Supplemental/Appendix (see below) 
\end{itemize}
\item Supplemental
\begin{itemize}
  \item See "Create a data collection plan" and "Create a data analysis plan" above
  \item Decide which figures and tables to include in your manuscript. A rule of thumb for an "average" manuscript might be 3 figures and 2 tables. Extra supporting information can go into Supplemental/Appendix (see below) 
\end{itemize}

\end{itemize}


\begin{table}[h]
\centering
\begin{tabular}{p{2cm}|p{4cm}|p{9cm}}

\textbf{Date} & \textbf{Lecture} & \textbf{Learning Outcomes}  \\ \hline

Sep 8 & Intro to R & \tabit Learn R syntax, objects, and basic plotting \tabit Custom functions \tabit Write simple R programs  \\ \hline

Sep 15 & Tidyverse: \texttt{dplyr} \& \texttt{ggplot2}  &  \tabit Principles of graphic design \tabit Introduction to the \texttt{tidyverse} \tabit Data wrangling, filtering, and organization \\ \hline

Sep 22 & Linear Models & \tabit Basic structure and terminology of linear models \tabit Effect sizes, model selection, partial effects plots \tabit Checking model results and output \\ \hline

Sep 29 & Generalized Linear Models (GLMs) & \tabit Common non-normal distributions \tabit GLM fitting and plotting \tabit Model validation, model selection for GLMs \tabit Preliminary models of your own data  \\ \hline

Oct 6 & Mixed effects models & \tabit Random versus fixed effects \tabit Random intercept and slope models \tabit Slope/intercept covariance, hypothesis testing \tabit Plotting of mixed models \\ \hline

Oct 13 & Nonlinear \& Additive models (GAMs) & \tabit Fitting strategies \tabit Generalized additive models (GAMs/``wiggly" models)  \tabit Distributional (non-stationary) models \\ \hline

Oct 20 & Spatiotemporal \& Dynamic models & \tabit Spatial and temporal random effects \tabit Dynamic models (e.g. logistic growth) \\ \hline

Oct 27 & Other topics & \tabit Multivariate models (e.g. community ordination) \tabit R as a GIS (e.g. mapping) \tabit Custom model coding (TMB or Stan) \\ \hline

Nov 3 & Writing & \tabit Structure of scientific papers (IMRaD) \tabit Writing clearly about models \tabit Reading about models critically \\ \hline

Nov 10 & Open work time & \tabit Time for open work on your own models and data \tabit Can work together/ask for help or clarification \\ \hline

Nov 17 & Reading break & Reading break \\ \hline

Nov 24 & Peer review & \tabit Draft write-up due \tabit Show us some of your results! \\ \hline

Dec 1 & Peer review & \tabit Reviews due \\ \hline

Dec 8 & Presentations & \tabit Final presentations \tabit Write-up due \\ 

\end{tabular}
\end{table}

\end{document}
