\documentclass[11pt]{article}
\usepackage{graphicx} % Required for inserting images
% \usepackage{multicol} %Multiple columns/rows in tables
\usepackage{geometry}
\geometry{
  top=0.5in,
  bottom=1in,
  left=1in,
  right=1in
}

\newcommand{\tabit}{\scriptsize\par\textbullet\phantom{ }}

\renewcommand{\familydefault}{\sfdefault}

\pagenumbering{gobble}

\title{BIOL 633: Term Project}
\author{Dr. Samuel V.J. Robinson}
\date{Fall 2023}

\begin{document}
\maketitle

\section*{Preamble}

\large
At this point in the term, we have learned about some common statistical approaches to different problems in ecology and evolution, and will now turn our efforts to putting this into practice. In this section we will gain experience in: 

\begin{itemize}
\item Creating \emph{research proposals} or \emph{IMRaD-style manuscripts} 
\item Providing critical, useful feedback in the form of a mock peer-review
\item Responding to feedback and further refining your manuscripts or proposals
\item Presenting your research to an audience of peers in a conference-style presentation
\end{itemize}

Doing statistics and data analysis as a working scientist is not simply a matter of fitting the ``correct" model or set of analyses and then pasting your ensuing results into a document. Like most academic disciplines, it is a matter of convincing your audience using a combination of word-smithing, modeling framework or mathematics (where needed), and graphic design. Even the most brilliant data collection and analysis is worth little if nobody can read or understand it! However, a lucid writing style and attractive figures are (hopefully) not everything: you must also convince your audience that the data you collected is \emph{a}) appropriate to answer your particular question and that \emph{b}) your methods of analysis were appropriate.

Depending on the timing of your particular degree program, you may choose to write either a \emph{proposal} or a \emph{manuscript}. If you've recently submitted your proposal, you may choose to do a more analytical version of it, but I would encourage you to be as forward-looking as possible, as the goal of this assignment is ultimately to produce material that can be used to further your scientific career.

The tasks that are before you:

\begin{enumerate}
\item Submit a manuscript or proposal, along with a cover letter to the ``editor" (SR)
\item Provide two anonymous\footnote{This will be semi-anonymous, given that the class is small. However, this is similar to "real" peer review: the pool of peer-reviewers in your field tends to be fairly small, so it's common to have reviewers that you partially know.} peer reviews for drafts from your colleagues
\item Write a formal response to the comments from your reviewers and editor, and re-submit the updated draft
\item Create a 15 minute conference presentation detailing either your proposal, or your submitted manuscript (13 minute presentation, 2 minutes for questions)
\end{enumerate}

\section*{Types of submissions}

\subsection*{Research proposal}

In general, a \textbf{proposal} must:
\begin{itemize}
  \item Explain and justify your research topic
  \begin{itemize}
    \item Introduce the reader to your research topic, and place it within the overall scope of your discipline
    \item Reveal the current problems or gaps in understanding. What is not known? What questions have other authors posed that are unanswered (or partly answered)?
    \item Explain why this problem matters. Is the answer to this problem theoretically or practically important? 
    \item Introduce your research goals to the reader in relation to the research gaps you've identified. Often this can be framed as your \emph{hypothesis}, or put in another way, "how you think the world works". What are the causal agents you identify?
    \item Make \emph{predictions} about what you're likely to find. If you find a certain results (positive, negative, neutral), will this support your original hypothesis, or lend support to an alternative one? Simple figures are often useful for this.
  \end{itemize}
  \item Methods: data collection
  \begin{itemize}
    \item Types of experimental controls/treatments that you plan to use
    \item Possible field work or laboratory setups
    \item Other sources of data (e.g. a literature review in the case of a meta-analysis)
    \item Identify what permissions or ethics approvals are required
  \end{itemize}
  \item Methods: data analysis
  \begin{itemize}
    \item What types of models or analysis will you use to answer your questions?
    \item Of the data that you collect, what are the \emph{dependent} and \emph{independent} variables?
    \begin{itemize} 
      \item If you're dealing primarily with observational data, defend your causal assumptions (i.e. why does X cause Y, rather than Y causing X?)
    \end{itemize}
    \item How are your models structured? R model formulas or mathematical structures can be helpful
    \item What parameters or other output will you use to identify support for your hypotheses? What types of tests will you use to identify strong ("significant") effects or negligible ones?
    \item How might these results look in a figure? This can be related to your earlier "predictions" figures
  \end{itemize}
  \item Identify an action plan, potential collaborators, timeline, and budget (\emph{not required for this assignment})
\end{itemize}

\subsection*{Manuscript}

An IMRaD (Introduction, Methods, Results, and Discussion) manuscript will be similar to the proposal in many respects, in particular with reference to introducing and justifying your research topic and approach, but in this case will be referring to things that you've already done. These manuscripts follow this general form:

\begin{itemize}
\item Abstract
\begin{itemize}
  \item A summary of the contents of the manuscript, including a brief summary of research goals, methods, results, and broader implications
\end{itemize}
\item Introduction
\begin{itemize}
  \item See "Explain and justify your research topic" above
\end{itemize}
\item Methods
\begin{itemize}
  \item See "Create a data collection plan" and "Create a data analysis plan" above
\end{itemize}
\item Results
\begin{itemize}
  \item Describe the data you collected, results of your analysis, and how it relates to your original hypothesis and predictions 
  \item Use figures or tables for supporting these results
\end{itemize}
\item Discussion
\begin{itemize}
  \item Relate your research findings back to the broader field. What new things has your research revealed?
  \item Briefly: future research or practical implications
\end{itemize}
\item Supplemental
\begin{itemize}
  \item Figures or Tables that are not central to the overall message of your paper (but still may be of interest to an interested reader) can go into a Supplemental.
  \item Small case-studies or pilot studies that were not large enough to warrant being included in the text, but that still influenced your research methods or conclusions, can also be included.
  \item R code (or other languages) is sometimes included here, if a specific program/algorithm was written for your analysis
\end{itemize}
\end{itemize}

\section*{Style}

\begin{itemize}
  \item Jargon and acronyms should be kept to a minimum. Both of these can help to quickly convey information, but only if the reader is also familiar with them. Some (e.g. p-values, ANOVA) will be common to all fields and won't require explanation, but this is somewhat flexible
  \item Make sure your submission uses standard English grammar. Spelling styles can be American or British, as long as they are consistent
  \item The document should be structured and well-organized, using section headings, subsections, and paragraphs with proper topic sentences. Ideas should flow between sections, paragraphs, and sentences in a logical way.
  \item Avoid passive voice, overly-complicated sentence structures, or wordiness. Include statistics (e.g. p-values) where needed, but not at the expense of the text. Aim for maximum readability and clarity
  \item Use figures and tables to back up your writing (and reduce the amount of text needed). Figure and table captions should stand alone, and not require other parts of the text in order to make sense.
  \item Decide which figures and tables are important to include in your text, and put the rest into a Supplemental/Appendix. A rule of thumb for an "average" manuscript is about two figures and three tables 
  \item Citation styles are up to the author, as long as the style is consistent
  
\end{itemize}

\newpage

\section*{Peer review expectations}

Peer review will follow a specs-grading framework, where each specification will be marked as 0 (incomplete) or 1 (complete). Bolded text indicates the overall specification, and further explanation/suggestion is provided in bulleted text. Both the peer reviewers and the "editor" will fill out a \textbf{a)} specs grading sheet and \textbf{b)} provide line-by-line comments, as described below

\begin{table}[h!]
\centering
\begin{tabular}{p{2cm}|p{13cm}}

\textbf{Section} & \textbf{Requirements}  \\ \hline

Cover letter & \textbf{A 1-page cover letter must be submitted, describing the general problem, methods, results, and implications} \tabit A brief statement about why the journal should be interested in your work is also common (i.e. will anyone who reads this journal be interested in this paper?) \tabit Cover letters can be based on the Abstract, but should not be a copy-paste of the Abstract. In general they should be addressed to the lead editor (or editorial board) \\ \hline

Abstract & \textbf{Manuscript is clearly and concisely summarized in a 300-word abstract} \\ \hline

Introduction & \textbf{Research topic is concisely introduced} \tabit Scope of research and knowledge gaps are identified \tabit Theoretically or practically relevance is justified \tabit Research questions, hypotheses, and predictions are stated \\ \hline

Methods: Data collection & \textbf{Data collection, including field work protocols or lab setups, is clearly explained} \tabit Sample size (or \emph{proposed} sample size) is stated, along with the sampling structure (e.g. how many samples per day/site) \\ \hline

Methods: Data analysis & \textbf{Analytical techniques are outlined and justified}
\tabit What are the \emph{dependent} and \emph{independent} variables? \tabit What types of models or analysis will you use, and how are they structured? R model formulas or mathematical structures can be helpful \tabit What parameters or other output will you use to identify support for your hypotheses? What types of tests will you use to identify strong ("significant") effects or negligible ones? \\ \hline

% \multicolumn{3}{c}{\textbf{IMRaD Manuscript Only}} \\ \hline

Results \small\emph{(IMRaD Manuscript Only)} & \textbf{The collected data are briefly described, and the results of your statistical analysis are clearly explained} \tabit Results are related back to the original hypotheses and predictions (but not expounded until the Discussion) \tabit Figures and tables are organized, easy to understand, and the author uses them to effectively communicate the results. Supporting figures are contained in the supplemental. Captions are clearly written, and make the figures and tables stand on their own \\ \hline

Discussion \small\emph{(IMRaD Manuscript Only)} & \textbf{Results are briefly summarized, and then related back to the broader field}
\tabit New findings are identified, and synthesized with existing research, with reference to specific papers where needed
\tabit Future research or practical implications are briefly discussed \\ \hline

Simulated Results \small\emph{(Research Proposal Only)} & \textbf{Simulated data \textbf{OR} a public dataset \emph{similar} to the one you will collect is described and statistically analyzed}
\tabit The data should be briefly described, and the results of your statistical analysis are clearly explained, along with appropriate figures and tables.
\tabit If data are generated, code is included in the supplemental. Possible sources of public data: Dryad, GBIF, iNaturalist  
\tabit Caveats for future analysis are briefly discussed. How might the future data (and results) differ from the ones analyzed here?  

\\ \hline

Writing & \textbf{Document is organized, well-written, and readable by non-experts}
  \tabit The document is submitted with \textbf{numbered lines} and \textbf{numbered pages}, along with standard page margins and font sizes. 
  \tabit The document is well-organized, with hierarchical sections, subsections, and paragraphs. Writing is clear and concise, with jargon and acronyms kept to a minimum.
  \tabit Standard English grammar and spellings are used. Passive voice is avoided, and sentences are not overly-complicated or wordy. Citation style is consistent and an appropriate number of citations are used.
  \tabit An appropriate number of figures and tables are contained in the text, and are used to effectively back up the writing \\

\end{tabular}
\end{table}

\subsection*{Written comments}

Comments are usually addressed directly to the paper authors, and can use the following format as a guideline:

\begin{itemize}
  \item Brief (single paragraph) description of the manuscript, restating the main goals, methods, and results from the authors, but in your own words (demonstrates that you understand the goal of the manuscript)
  \item Statement of the main successes and potential flaws of the manuscript, as well as a statement about the type of revisions that you'd recommend (\emph{accept, minor revisions, major revisions, reject}: the editor will have the final word on which one the author will receive).
  \item Major comments: are there any big overall things that you'd recommend to the author (e.g. \emph{I would exclude this type of treatment because...}, \emph{This type of test should be changed to...}). If you have any other papers or texts the authors would benefit from, cite them here (but avoid self-serving citations).
  \item Line-by-line comments: for comments on specific lines, provide a list of comments for each line you recommend changes on (e.g. \emph{L40: Change the spelling to ...}, \emph{L165: How did you test this claim? Do you have a p-value or other model statistic?}, \emph{This approach was also used by Smith et al. 2010, but you found different results. Why might that be?}) 
  \item Remember to provide \emph{potential solutions} to any issues that you've brought up: \emph{review others as you would wish to be reviewed}!
  
\end{itemize}

\newpage

\section*{Peer review grading}

Reviews will be marked based on the following (fairly simple) specifications:

\begin{table}[h!]
\centering
\begin{tabular}{p{3cm}|p{12cm}}
\textbf{Section} & \textbf{Requirements} \\ \hline
Specs grade & \textbf{Reviewer provides specs grading marks for each review} \\ \hline
Written comments & \textbf{Reviewer provides written feedback for each review} \tabit Feedback must be structured, clear, and provide helpful suggestions (summary, major comments, and line-by-line) \\ 
\end{tabular}
\end{table}

\section*{Revisions}

Responding to revisions is a large part of the scientific writing process, either from your colleagues, supervisor, or peer reviewers. Responses to reviewers are assembled into a single document that responds to the critiques brought forth by the editor and the reviewers, and will be marked as follows:

\begin{table}[h!]
\centering
\begin{tabular}{p{3cm}|p{12cm}}
\textbf{Section} & \textbf{Requirements} \\ \hline

Response to major comments & \textbf{Author responds to major comments from editor and reviewers} \tabit Corrections must be specific to comments from each reviewer/editor, should demonstrate what changes you've made, and show why this addresses the problem identified \tabit Rebuttals can also be made (i.e. \emph{The reviewer doesn't understand this...}), but this is usually an indication of unclear explanations. If the reviewer is wrong, then you can defend yourself, but make sure that your corrections account for this misunderstanding and preempt future readers \\ \hline
Response to line-by-line & \textbf{Author responds to minor (line-by-line) comments from editor and reviewers} \tabit Corrections must address specific comments from each reviewer/editor \\

\end{tabular}
\end{table}

Good writing is re-writing. Because of this, incomplete marks from the first draft will be upgraded to complete \emph{if you make the corrections recommended by the editor and reviewers}. 

\section*{Final presentation}

TBA

\end{document}
